%!TEX TS-program = xelatex
% https://tex.stackexchange.com/questions/325278/xelatex-runs-slow-on-windows-machine
\documentclass[]{friggeri-cv}
\makeauthorbold{Byun}
\addbibresource{bibliography.bib}

\begin{document}
\header{Jung-Hyun }{Byun}
       {Last updated: December 08, 2018}

% In the aside, each new line forces a line break
\begin{aside}
  \section{contact}
    \href{mailto:junghyun.ian.byun@gmail.com}{junghyun.ian.byun\\@gmail.com}
%     \href{https://github.com/IanByun}{[GitHub]}
%     \href{https://scholar.google.co.kr/citations?user=JzneZIcAAAAJ}{[Scholar]}
%     \href{https://orcid.org/0000-0002-1248-292X}{[ORCID]}
%     \href{https://www.linkedin.com/in/junghyun-ian-byun/}{[LinkedIn]}
  \section{languages}
    Korean (native)
    English (fluent)
  \section{programming}
    C++ (skilled)
    Python/CUDA/Matlab/\\Java (beginner)
  \section{skills}
    OpenCV, OpenGL, openFrameworks
\end{aside}

%\section{About}

\section{Interests}

computer vision, computer graphics, machine learning and human-computer interaction\\
augmented reality, projection mapping, point cloud processing and scene reconstruction

\section{Education}

\begin{entrylist}
  \entry
    {2015.9.1\\--Current}
    {Ph.D. {\normalfont course in Computer Science}}
    {Yonsei University, Korea}
    {}
  \entry
    {2011.3.1\\--2015.2.28}
    {B.Sc. {\normalfont in Computer Science and Engineering}}
    {Yonsei University, Korea}
    {}
\end{entrylist}

\section{Selected Publications}

\begin{refsection}
  \nocite{*}
  \printbibliography[
    type=article, 
    title=\textbf{Publications of peer-reviewed journal articles}, 
    heading=subbibliography
  ]
\end{refsection}

\begin{refsection}
  \nocite{*}
  \printbibliography[
    type=inproceedings, 
    title=\textbf{Proceedings of peer-reviewed conference papers}, 
    heading=subbibliography,
    keyword={selected}
  ]
\end{refsection}

\section{Awards}

\begin{entrylist}
  \entry
    {2018}
    {Best Demo Award}
    {}
    {ACM International Conference on Multimedia (ACM MM)}
\entry
    {2018}
    {Ph.D. Fellowship Award}
    {}
    {NAVER Corporation}
\end{entrylist}

\section{Projects}

\begin{entrylist}
  \entry
    {2018.09.01\\--2020.08.31} %
    {Integration of Context-aware Pervasive AR Platform for Personal Assistant Implementation}
    {National Research Foundation (NRF), 266K USD/year}
    {Role: Project Manager \& Lead Researcher\\
    \\
    $\bullet$ Research on applicability of deep learning-based spatial context-awareness in an augmented reality environment.\\
    $\bullet$ Research on integration of  scene understanding technologies with projection-based augmented reality.\\
    $\bullet$ Research on real-time dynamic projection mapping on a pan-tilt platform.
    }
  \entry
    {2018.04.30\\--2018.10.31}
    {Development of hand motion recognition technology based on sensor fusion}
    {Samsung Electronics Company, 48K USD/year}
    {Role: Project Manager\\
    
    $\bullet$ Managed implementation of algorithms for identifying hand postures of workers using IMU sensor data.
    }
  \entry
    {2015.11.01\\--2018.10.31} 
    {Pervasive AR interaction platform construction using a mobile projection technology}
    {National Research Foundation (NRF), 264K USD/year}
    {Role: Project Manager \& Lead Researcher\\
    \\
    $\bullet$ Designed a user-perspective rendering algorithm for correcting distortions of projection mapping caused by surface geometry.\\
    $\bullet$ Designed  a visual servoing algorithm for accurately controlling pan-tilt servo motors based on rotation axis calibration.
    }
  \entry
    {2015.08.01\\--2017.03.31}
    {Development of filming and rendering technology based on multi-autonomous flight collaboration for large-scale performance and broadcasting}
    {Korea Institute of Science and Technology (KIST), 26K USD/year}
    {Role: Researcher \& Developer\\
    \\
    $\bullet$ Designed and implemented scale-adaptive visual object tracking algorithm based on SVM.\\
    $\bullet$ Developed a Windows program for tracking multiple objects based on epipolar geometry.
    }
  \entry
    {2015.04.01\\--2017.12.31}
    {Research of vision-based mobile object recognition technology for life logging}
    {Korea Institute of Science and Technology (KIST), 44K USD/year}
    {Role: Researcher \& Developer\\
    \\
    $\bullet$ Implemented keypoint extraction and descriptor matching algorithms on an Android platform.\\
    $\bullet$ Developed Android applications for marker-less augmented reality and medicine recognition.
    }
\end{entrylist}

\section{Other Publications}

\begin{refsection}
  \nocite{*}
  \printbibliography[
    type=inproceedings, 
    title=\textbf{Proceedings of peer-reviewed conference papers}, 
    heading=subbibliography,
    notkeyword={selected}
  ]
\end{refsection}

\end{document}
