%!TEX TS-program = xelatex
% https://tex.stackexchange.com/questions/325278/xelatex-runs-slow-on-windows-machine
\def\cvprint{This changes color presets.}
\documentclass[]{friggeri-cv}
\makeauthorbold{Byun}
\addbibresource{bibliography.bib}
%https://tex.stackexchange.com/questions/246524/remove-asterik-in-friggeri-template
\defbibheading{subbibliography}[\refname]{\subsection{#1}}

\usepackage{kotex}
\setmainhangulfont{나눔바른고딕}

\begin{document}
\header{Jung-Hyun }{Byun}
       { 변정현, Contact:
       \href{mailto:junghyun.byun@yonsei.ac.kr}{junghyun.byun@yonsei.ac.kr}}

% In the aside, each new line forces a line break
\begin{aside}
  \section{currently at}
    Ph.D. Candidate, Yonsei University, Seoul, Korea
%     \href{https://github.com/IanByun}{[GitHub]}
%     \href{https://scholar.google.co.kr/citations?user=JzneZIcAAAAJ}{[Scholar]}
%     \href{https://orcid.org/0000-0002-1248-292X}{[ORCID]}
%     \href{https://www.linkedin.com/in/junghyun-ian-byun/}{[LinkedIn]}
  \section{languages}
    Korean (native)
    English (fluent)
  \section{programming}
    C++ (skilled)
    Python/CUDA/\\Matlab/Java (user)
  \section{skills}
    OpenCV, OpenGL, openFrameworks
  \section{last updated}
    June 10, 2020
  \section{view online}
    \includegraphics[width=\textwidth]{data/cv_qr_code}
\end{aside}

%-----------------------------------------------------------------------------------------------------------------
%\section{About}

%-----------------------------------------------------------------------------------------------------------------
\section{Interests}

computer vision, computer graphics, machine learning and human-computer interaction\\
augmented reality, projection mapping, point cloud processing and scene reconstruction

%\par\null%\par\null\par
%-----------------------------------------------------------------------------------------------------------------
\section{Education}

\begin{entrylist}
  \entry
    {2015.9.1\\--2020.08.31}
    {Ph.D. {\normalfont candidate in Computer Science}}
    {Yonsei University, Seoul, Korea}
    {Thesis: Projection Mapping and Augmented Reality for Pervasive AR \\Framework and Environment}
  \entry
    {2011.3.1\\--2015.2.28}
    {B.Sc. {\normalfont in Computer Science and Engineering}}
    {Yonsei University, Seoul, Korea}
    {}
\end{entrylist}

\par\null%\par\null\par
%-----------------------------------------------------------------------------------------------------------------
\section{Selected Publications}

\begin{refsection}
  \nocite{*}
  \printbibliography[
    type=article, 
    title=\textbf{Journal articles}, 
    heading=subbibliography,
    keyword={selected}
  ]
\end{refsection}

\begin{refsection}
  \nocite{*}
  \printbibliography[
    type=inproceedings, 
    title=\textbf{Conference proceedings}, 
    heading=subbibliography,
    keyword={selected}
  ]
\end{refsection}

\par\null%\par\null\par
%-----------------------------------------------------------------------------------------------------------------
\section{Awards}

\begin{entrylist}
\entry
    {2019}
    {Merit Academic Paper Award (우수논문 장려상)}
    {}
    {Yonsei University \hangulfontspec{나눔고딕}(연세대학교)}
\entry
    {2019}
    {Best Paper Presentation Award (우수 논문 발표상)}
    {}
    {Korea Multimedia Society \hangulfontspec{나눔고딕}(한국멀티미디어학회)}
\entry
    {2018}
    {Ph.D. Fellowship Award}
    {}
    {NAVER Corporation \hangulfontspec{나눔고딕}(네이버 주식회사)}
\entry
    {2018}
    {Best Demo Award}
    {}
    {ACM International Conference on Multimedia (ACM MM)}
\end{entrylist}

%-----------------------------------------------------------------------------------------------------------------
\section{Invited Talks}

\begin{entrylist}
\entry
    {2019}
    {NAVER Tech Talk}
    {NAVER Corporation \hangulfontspec{나눔고딕}(네이버 주식회사)}
    {Projection Mapping and Augmented Reality for Pervasive AR Environment}
\end{entrylist}

%-----------------------------------------------------------------------------------------------------------------
\section{Patent}

\begin{refsection}
  \nocite{*}
  \printbibliography[
    type=misc, 
    title=\textbf{Domestic (Republic of Korea)}, 
    heading=subbibliography,
    keyword={patent}
  ]
\end{refsection}

%-----------------------------------------------------------------------------------------------------------------
\section{Projects}

\begin{entrylist}
  \entry
    {2018.09.01\\--2020.08.31} %
    {퍼스널 어시스턴트 구현을 위한 맥락인지 Pervasive AR 플랫폼 구축\\
    Integration of Context-aware Pervasive AR Platform for Personal Assistant Implementation}
    {National Research Foundation of Korea \hangulfontspec{나눔고딕}(한국연구재단)}
    {Role: Project Manager \& Lead Researcher\\
    \\
    $\bullet$ Research on applicability of deep learning-based spatial context-awareness in an augmented reality environment.\\
    $\bullet$ Research on integration of  scene understanding technologies with projection-based augmented reality.\\
    $\bullet$ Research on real-time dynamic projection mapping on a pan-tilt platform.
    }
  \entry
    {2018.04.30\\--2018.10.31}
    {센서 융합 기반 손 동작 인식 기술 개발\\
    Development of hand motion recognition technology based on sensor fusion}
    {Samsung Electronics Company \hangulfontspec{나눔고딕}(삼성전자)}
    {Role: Project Manager\\
    
    $\bullet$ Managed implementation of algorithms for identifying hand postures of workers using IMU sensor data.
    }
  \entry
    {2015.11.01\\--2018.10.31} 
    {이동형 프로젝션 기술을 이용한 Pervasive AR 인터랙션 플랫폼 구축\\
    Pervasive AR interaction platform construction using a mobile projection technology}
    {National Research Foundation of Korea \hangulfontspec{나눔고딕}(한국연구재단)}
    {Role: Project Manager \& Lead Researcher\\
    \\
    $\bullet$ Designed a user-perspective rendering algorithm for correcting distortions of projection mapping caused by surface geometry.\\
    $\bullet$ Designed  a visual servoing algorithm for accurately controlling pan-tilt servo motors based on rotation axis calibration.
    }
  \entry
    {2015.08.01\\--2017.03.31}
    {대규모 공연 및 방송을 위한 다중 자율 비행체 협업 기반 첨단 촬영 및 렌더링 기술 개발\\
    Development of filming and rendering technology based on multi-autonomous flight collaboration for large-scale performance and broadcasting}
    {Korea Institute of Science and Technology \hangulfontspec{나눔고딕}(KIST, 한국과학기술연구원)}
    {Role: Researcher \& Developer\\
    \\
    $\bullet$ Designed and implemented scale-adaptive visual object tracking algorithm based on SVM.\\
    $\bullet$ Developed a Windows program for tracking multiple objects based on epipolar geometry.
    }
  \entry
    {2015.04.01\\--2017.12.31}
    {라이프 로깅을 위한 영상 기반 모바일 객체 인식 연구 개발\\
    Research of vision-based mobile object recognition technology for life logging}
    {Korea Institute of Science and Technology \hangulfontspec{나눔고딕}(KIST, 한국과학기술연구원)}
    {Role: Researcher \& Developer\\
    \\
    $\bullet$ Implemented keypoint extraction and descriptor matching algorithms on an Android platform.\\
    $\bullet$ Developed Android applications for marker-less augmented reality and medicine recognition.
    }
\end{entrylist}

%-----------------------------------------------------------------------------------------------------------------
\section{Other Publications}

\begin{refsection}
  \nocite{*}
  \printbibliography[
    type=article, 
    title=\textbf{Journal articles}, 
    heading=subbibliography,
    notkeyword={selected}
  ]
\end{refsection}

\begin{refsection}
  \nocite{*}
  \printbibliography[
    type=inproceedings, 
    title=\textbf{Conference proceedings}, 
    heading=subbibliography,
    notkeyword={selected}
  ]
\end{refsection}

\end{document}
